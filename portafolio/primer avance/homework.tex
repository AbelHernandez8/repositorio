\documentclass{article}

\usepackage[english]{babel}

\title{Homework}
\author{Hernandez De Lara Abel}

\begin{document}
\maketitle

\section{PWA CONCEPTS AND FEATURES}
\subsection{What are they?}
PWAs are commonly defined as Apps that bring together the best of web and native applications, even being understood as a middle ground or an evolved form.
They are based on web pages, but use technologies that make their aesthetics and operation very similar to a native App. They are accessed through a browser, but a direct access can be anchored in our device. They do not depend on operating systems and incorporate native functionalities of the device.

\subsection{Features}
\begin{itemize}


\item Responsive:
Currently, most websites have responsive design that allows them to adapt to different devices.

\item Updated:
PWAs will always show their latest version to the user, with the use of automatic updates. 

\item Secure:
The HTTPS secure protocol is always used, which is also necessary for the installation of the Service Worker. 

\item Fast:
Generally, a PWA is optimized for both loading and browsing speed. 


\section{DIFFERENCE BETWEEN PROGRESSIVE AND SERVICE-ORIENTED WEB APPS }

PWAs are web applications that use APIs and emerging web browser functions along with a traditional incremental enhancement strategy to deliver a native application on the other hand, a service-oriented application focuses on service architecture, dividing functionalities into independent components that communicate with each other. 

\section{ADVANTAGES AND DISADVANTAGES OF PWA}


Advantages:
\item Low cost
\item Requires fewer resources
\item They are secure
\item Always up to date 
\item Offline access 

\vspace{12pt}

Disadvantages:
\item They do not appear in the app stores
\item They consume a lot of battery 
\item Limited use of native functionalities
\item Audience is used to native Apps


\section{DEVELOPMENT AND IMPLEMENTATION TOOLS FOR PWA}

\subsection{Requirements}
\item A site must be visited twice with an interval of 5 minutes to qualify: It is not the most reliable way of verification, but it is a simple way to determine the user's interest.
\item Valid secure HTTPS connection: By having a secure connection to the progressive web app, users can feel relatively safe allowing permissions to the PWA.
\item Valid JSON Manifest installed: By providing a data extract in JSON format, it is possible to cache this information with the help of the service worker. 
\item Service Worker installed: As we said, the service worker is responsible for caching all files, push notifications, content update, data manipulation.

\subsection{Better environment}
Flutter, Google's new environment, is a good example of technology that adapts perfectly to PWA development, achieving native performance and cross-platform development.

\end{itemize}
\end{document}