\documentclass{article}

\begin{document}

\section*{Estructura de la Tarea}

\subsection*{Parte 1: Introducción}
La Universidad Tecnológica de Tijuana enfrenta desafíos en la gestión de estacionamientos, con la necesidad de optimizar el ingreso de alumnos y supervisar la disponibilidad de espacios. UPARK surge como una solución para abordar estos desafíos, mejorando la eficiencia y seguridad en el manejo del estacionamiento universitario.

\subsection*{Parte 2: Objetivos y Propósitos}
\subsubsection*{Objetivos}
\begin{itemize}
    \item Optimización del Ingreso de Alumnos: Facilitar el registro y acceso de alumnos al estacionamiento, proporcionando información en tiempo real sobre la disponibilidad de espacios.
    \item Mejora de la Seguridad: Implementación de un sistema eficiente para registrar y supervisar el tráfico en el estacionamiento.
\end{itemize}

\subsubsection*{Propósitos}
\begin{itemize}
    \item Mejorar la Experiencia del Usuario: Proporcionar una experiencia de estacionamiento eficiente, rápida y conveniente.
    \item Contribuir a un Ambiente Universitario más Seguro y Ordenado: Gestionar eficientemente el flujo de vehículos en el campus.
    \item Minimizar los Tiempos de Espera y Congestionamientos: Reducir tiempos de espera y evitar congestiones en el área de estacionamiento.
\end{itemize}

\subsection*{Parte 3: ¿Qué son las PWA?}
Las Progressive Web Apps (PWA) son una forma innovadora de desarrollar aplicaciones web que combinan lo mejor de las aplicaciones móviles y las páginas web tradicionales.

\subsubsection*{Características Claves}
\begin{itemize}
    \item Acceso Offline.
    \item Experiencia del Usuario Mejorada.
    \item Seguridad.
    \item Adaptabilidad a Múltiples Dispositivos.
\end{itemize}

\subsection*{Parte 4: Razones para Elegir PWA en UPARK}
\begin{itemize}
    \item Costos de Desarrollo y Mantenimiento Reducidos.
    \item Compatibilidad con Diferentes Navegadores.
    \item Retroalimentación Continua y Mejora Iterativa.
\end{itemize}

\subsection*{Parte 5: ¿Cómo se Mejora la Experiencia del Usuario en UPARK Mediante las PWA?}
\begin{itemize}
    \item Navegación Rápida.
    \item Interactividad Mejorada.
    \item Capacidades Offline sin Compromisos.
\end{itemize}

\subsection*{Parte 6: Conclusión}
En resumen, la elección de desarrollar UPARK como Progressive Web App (PWA) se sustenta en razones fundamentales que buscan optimizar la gestión de estacionamientos en la Universidad Tecnológica de Tijuana (UTT). Las razones clave que respaldan esta elección incluyen:
\begin{itemize}
    \item Instalación sin Fricciones y Actualizaciones Automáticas.
    \item Eficiencia en el Desarrollo y Mantenimiento.
    \item Seguridad y Adaptabilidad.
\end{itemize}

\end{document}
